\documentclass[twocolumn]{article}
% \documentclass{article}

\usepackage[utf8]{inputenc}
\usepackage[french]{babel}

\newcommand{\hsp}{\hspace*{5mm}}
\newcommand{\tb}{\textbf}

\title{Cours d'italien}
\date{}

\begin{document}
 
\maketitle

\section{Présentation}

Quelques phrases utiles:\\

\begin{itemize}
 \item Come ti chiami ? \\
\hsp Mi chiamo Nicolas.\\
\hsp Sono Nicolas. (Nicolas è il mio nome)
 \item Qual \tb{è} il tuo/suo cognome ?\\
\hspace*{5mm} Il mio cognome è Perrin, \tb{e} tu ?\\
(\textbf{Lei} à la place de ``tu'' si on veut être plus formel ; remarque : ``lei'' sans majuscule veut dire ``elle'', et la conjugaison avec ``Lei'' se fait comme à la troisième personne du singulier : ainsi, la version formelle de ``Come ti chiami ?'' est ``Come \tb{si} chiama'')
 \item Io abito \tb{a} Genova. E tu, dove abiti ?\\
\hsp Abito a Genova, \tb{in} via San Lorenzo.
 \item Di dove sei ? (version formelle : Di dove è / Di dov'è.)\\
\hsp Sono di Genova, sono italiana.\\
\hsp Sono di Metz, sono francese.\\
\hsp Vengo dalla Francia.
 \item Sei americano ?\\
\hsp No, sono francese.
\end{itemize}

Un peu de vocabulaire:
\begin{itemize}
 \item réponse = riposta
 \item les salutations = i saluti
 \item enchanté = piacere
 \item bonjour = buongiorno (``buongiorno'' va bene fine alle quindici (15h), dopo le 3 diciamo ``buenasera'')
 \item bonjour = salve (formel)
 \item salut = ciao
 \item au revoir = buona giornata / arrivederci
\end{itemize}



\end{document}
